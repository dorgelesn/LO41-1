\documentclass{report}

\usepackage[francais]{babel}
\usepackage[utf8x]{inputenc}
\usepackage[T1]{fontenc}
\usepackage[final]{pdfpages}
\usepackage{graphicx}
\usepackage{array}
\usepackage{eurosym}
\usepackage{listings}
%\usepackage{gasetex}
\title{Université de Technologie Belfort-Montbéliard\\
Projet LO41\\
gestion des carrefours routiers}
\author{Lacour Florian\\
Michael Ayeng}

\lstdefinestyle{customc}{
  belowcaptionskip=1\baselineskip,
  xleftmargin=\parindent,
  language=C,
  showstringspaces=false,
  basicstyle=\footnotesize\ttfamily,
  keywordstyle=\bfseries\color{green!35!black},
  commentstyle=\itshape\color{purple!30!black},
  identifierstyle=\color{blue},
  stringstyle=\color{orange},
}

\lstset{escapechar=@,style=customc}

\begin{document}



\maketitle

\tableofcontents



\chapter{Communication inter Thread}
	\section{Les Semaphore}
		\paragraph{}
			Tous les thread utilisent des semaphore en effet ces semaphore permet de stocker le nombre de thread qui desire rentrer en section critique. Tous les semaphore sont initialiser dans le main, de ce fait il est neccessaire de decider d'un nombre maximum de voiture a gerer.

		\section{Echangeur}
			Chaque echangeur possede 2 semaphores, le premier semaphore permet de connaitre le nombre de voiture qui desire entrer dans le carrefour, le second semaphore est utiliser par la voiture pour libérer le carrefour.

		\section{Voiture}
			Les voiture posséde un semaphore afin de recevoir le signal de depart de la part de l'echangeur. 

	\section{Le Serveur}
		\paragraph{}
			Le Serveur possede un semaphore permettant de le prevenir lorsque une voiture demande a rentrer dans l'echangeur.
		
		
	

\end{document}
